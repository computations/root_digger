\documentclass{article}

\title{Root Digger: a root placement program for phylogenetic trees}
\author{Ben Bettisworth, Alexandros Stamatakis}
\date{\today}

\begin{document}
\begin{abstract}
In phylogenetic analysis, it is common to infer trees which are unrooted. This
is to say, it is unknown which node is the most recent common ancestor of all
the taxa present in the phylogenetic tree. This information is often desirable,
as it provides a direction to the edges in the tree.  There exist several
methods to recover a root, including midpoint rooting and using a special taxon
as a so called outgroup. Additionally, a non-reversable Markov model can be used
to compute the likelihood of a root. In this paper, we present software which
uses a provided non-reversable model to compute the most likely root location.
\end{abstract}

\maketitle

% Main Points:
% - Rerunning analysis with outgroups is expensive, and can introduce errors
% - Midpoint rooting is a joke
% - This is fast and accurate.


\section{Introduction}

% Outline:
% - What is rooting
% - Why do we need it
% - Existing methods
% - What this method is

\section{Background}

% Outline:
% - What a non-reversable model is
% - How it roots a tree

\section{The Software}

% Outline:
% - Options
% - How the software works (maybe)

\section{Experiment and Results}

% Outline:
% - indel data generation
% - run setup (cpu, memory etc)
% - show results (time, accuracy)
%   - Accuracy should be measured by getting the root on the right edge, as well
%   as how far along that edge the root is.

\section{Conclusion}

% Outline:
% - Tell 'em what you told 'em

\end{document}
