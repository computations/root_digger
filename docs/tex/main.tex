\documentclass{article}
\usepackage{scrextend}
\usepackage{lineno}
\usepackage{ctable}
\usepackage{multirow}
\usepackage{hyperref}
\usepackage{graphicx}
\usepackage{xcolor}
\usepackage{float}
\usepackage{setspace}
\usepackage[a4paper]{geometry}
\usepackage[numbers,sort]{natbib}

\title{Root Digger: a root placement program for phylogenetic trees}
\author{Ben Bettisworth, Alexandros Stamatakis}
\date{\today}

\newcommand{\RootDigger}{RootDigger}
\newcommand{\RootDiggertt}{\texttt{RootDigger}}
\newcommand{\BenComment}[1]{{\bf \color{blue} ({#1})}}
\newcommand{\AlexisComment}[1]{{\bf \color{green} ({#1})}}

\doublespace
\linenumbers

\begin{document}
\begin{abstract}
  In phylogenetic analysis, it is common to infer trees which are unrooted.
  This is to say, it is unknown which node is the most recent common ancestor of
  all the taxa present in the phylogenetic tree.  However, knowing the root
  location is often desirable, as it provides a direction to the edges in the
  tree.  There exist several methods to recover a root, including midpoint
  rooting and using a special taxon as a so called outgroup.  Additionally, a
  non-reversible Markov model can be used to compute the likelihood of a root.
  In this paper, we present software called \RootDiggertt{} which uses a
  provided non-reversible model to compute the most likely root location.
\end{abstract}

\maketitle

\section{Introduction}

When inferring phylogenetic trees, most phylogenetic inference tools
\cite{nguyen_iq-tree:_2015, stamatakis_raxml_2014, minh_iq-tree_2019} will infer
an unrooted tree. This is due to a standard assumption for phylogenetic tools:
reversibility of the character substitution process
\cite{felsenstein_evolutionary_1981}.  This yields the problem of inferring a
phylogenetic tree computationally tractable.  However, by making this assumption
the ability to identify a root is lost. A rooted phylogenetic tree is desirable,
though, as it can resolve long standing disputes regarding the placement of
large clades on the tree of life \cite{dunn_animal_2014} for example.

There exist, however, several methods to recover the root on an existing
phylogenetic tree, which generally fall into one of three categories: methods
which use additional topological information not present in the sequence data;
methods which utilize some variation of the molecular clock hypothesis; and
methods which utilize a non-reversible model.

Methods that use additional topological information take advantage of prior
knowledge about the world, which is not present in the data that is used to
infer the tree. In particular, knowledge about species which are distantly
related can be used to include a so-called outgroup to phylogenetic analysis.
This outgroup can then be used to place the root on the tree, as the most recent
common ancestor of the ingroup and the outgroup must be the root of the tree.

There are challenges to including an outgroup to an analysis.  Gatsey et. al.
\cite{gatesy_how_2007} showed how adding a single taxon to an analysis can
substantially change the resulting tree topology, even for the taxa which were
already present in the analysis before (i.e., the ingroup).  Holland
\cite{holland_outgroup_2003} investigates this phenomenon in simulations, and
finds that outgroups that affect or alter the topology of ingroups are common.

Alternatively, molecular clock analysis can be used to place a root without
prior topological knowledge \cite{yang_computational_2006}.  The molecular clock
hypothesis states that base substitution ``ticks'' at a stochastically constant
rate.  Using this assumption, or some variant thereof, a likely location can be
inferred for the root on an existing phylogenetic tree.  A simple version of
this is midpoint rooting, which relies on a constant molecular clock assumption.

Molecular clock analyses exhibit their own difficulties. In particular, the
clock does not "tick" at a constant rate over the tree as shown in
\cite{li_molecular_1987} and \cite{steiper_primate_2006}. Relaxed clock models
exist which can alleviate this problem, but are not always successful at
correctly identifying the root as shown in \cite{battistuzzi_performance_2010}.

The final method that can place a root on a tree is to perform the phylogenetic
analysis using a non-reversible model. By using a non-reversible model, we can
place a direction of time on the branches of a tree \cite{yap_rooting_2005}.
Several software packages are able to infer a phylogenetic tree under a
non-reversible model, and thereby identifying a root \cite{nguyen_iq-tree:_2015,
ronquist_mrbayes_2003}.

Non-reversible models for phylogenetic trees come in many forms.  For example,
accounting for duplication, transfer, and loss events yields a non-reversible
model \cite{morel_generax:_2019}.  In particular, duplication events have been
used for rooting trees \cite{emms_stride:_2017}.  Another method, the one
primarily used in this work, is to eliminate the reversibility assumption of
standard character (e.g., nucleotide or amino acid) substitution models.
Unfortunately, eliminating this assumption significantly increases the
computational effort required to find a good (high likelihood) phylogenetic
tree.  This is due to the loss of the pulley principle
\cite{felsenstein_evolutionary_1981}, which allows phylogenetic inference tools
to ignore root placement during tree inference.  Therefore, by adopting a
non-reversible model, the location of a root on a phylogenetic tree affects the
likelihood of that tree.

The location of the root affects the likelihood of the tree, thus using standard
tree search techniques all possible rootings would need to be evaluated for each
tree in order to find the rooting with the highest likelihood.  In the worst
case, this increases the work {\em per tree} by a factor of $\mathcal{O}(n)$
where $n$ is the number of taxa present in the dataset. Therefore, eliminating
the reversibility assumption drastically increases the computational effort
required to infer a tree. Hence, standard inference tools choose to adopt
the reversibility assumption, as phylogentic tree inference would be
computationally intractable otherwise.

As an alternative to the computationally expensive process of inferring a tree
with a root, an unrooted tree which has already been inferred under a reversible
model can be evaluated a posteriori for possible root locations under a
non-reversible model. This requires less computational effort, as it skips the
expensive step of looking for ``good'' rootings in intermediate trees during the
tree search.  With this method, we can find the most likely root location for a
given phylogenetic tree.  Even this method still has numerical challenges, as
the likelihood function for rooting a phylogenetic tree may exhibit several
local maxima \cite{huelsenbeck_inferring_2002}, although we did not find this to
be a major issue in our experiments (see the discussion).

We implemented the open source software tool \RootDiggertt{} which uses a
non-reversible model of character substitution to infer a root on an already
inferred tree. The inputs to our tool are a multiple sequence alignment (MSA)
and a phylogenetic tree. \RootDiggertt{} then yields a rooted tree.
\RootDiggertt{} implements fast and a slow root finding modes, called Search
mode and Exhaustive mode, respectively. The search mode simply finds the most
likely root quickly \AlexisComment{view appropriate identities}. The exhaustive
mode evaluates the likelihood of placing the root into every branch of the given
tree, and reports the likelihood weight ratio \cite{strimmer_inferring_2002} for
placing a root on that branch for every branch on the tree. In addition to the
two search modes, there is an optional early stop mode, which can be combined
with either root finding mode. In this early stop mode, the search will
terminate if the root placement is nearly the same twice in a row. This is to
say, if the location of the best root position is on the same branch as in the
previous iteration {\em and} the value inferred for the root position is
sufficiently close the current iteration terminates.  While the early stop
optimization does improve rooting times substantially (approximately a factor of
1.7), the likelihood of each root placement will not be fully optimized.  In
practice, this does not substantially affect the final root placement, but it
does render comparison of the likelihood with results from other tools invalid.

\section{Methods}

The input to \RootDiggertt{} is a MSA and a phylogenetic tree with branch
lengths in expected mean substitutions per site.  \RootDiggertt{} then uses the
tree and branch lengths to find the most likely root location by calculating the
likelihood of a root location under a non-reversible model of DNA substitution
(specifically, UNREST \cite{yang_estimating_1994} with a user specified number
of $\Gamma$ discrete rate categories).  The optimal position of the root along a
specific branch is calculated by splitting the given branch in two with
resulting branch lengths $\alpha t$ and $(1-\alpha) t$, with $0 \leq \alpha \leq
1.0$.  We then find the maximum likelihood value of $\alpha$, and report the
likelihood for the given branch as the likelihood of the root location.  By
formulating the problem this way, we can use single parameter optimization
techniques such as Brent's Method\footnotemark{}, which are computationally more
efficient compared to multi-parameter optimization routines.

\footnotetext{
  Brent's Method is picked over Newton's Method because it does not require a
  second derivative to optimize the function.

  We computed an analytical version of the second derivative, but the library
  that we use doesn't support calculating some of the values involved.
  The effort to rewrite the functions that would compute these values
  deemed to be too much relative to the savings.
  In principle, the computation could be done.
}

A potential problem of Brent's and analogous methods is that they find extrema
by identifying roots for the objective function's derivative. In order to find
maxima, though, it is required that the objective function's value is evaluated,
as a root of the derivative could correspond to a minima. In addition, Brent's
method will fail to find all extrema. To alleviate, we need to search for
bracketing windows that can be used to safely find extrema.  Unfortunately, we
are not aware of a general method for finding such bracketing windows, so a
recursive method is employed, were the search range is reduced in half and
adequately searched for appropriate\footnotemark{} windows.

\footnotetext{Appropriate here means that the sign of the function in question
  has opposite signs at the endpoints of the window.}

As mentioned in the introduction, \RootDiggertt{} has 2 modes of operation.
These modes will be discussed individually, starting with search mode:

\begin{enumerate}
  \item Initialize numerical model parameters:
        \begin{itemize}
          \item $\alpha$-shape parameter for discrete $\Gamma$ rates to $1.0$
            (if applicable),
          \item Character substitution rates to uniform random values according
                to $U(0.0001, 1.0)$,
          \item Base frequencies to uniform random values according to
                $U(0.0001, 1.0)$, which are then normalized such that the sum is
                equal to 1.
        \end{itemize}
  \item Choose starting roots by best log-likelihood at each branch midpoint
    (default 5\%).
  \item For each starting root:
        \begin{enumerate}
          \item Optimize model parameters
                \begin{itemize}
                  \item $\alpha$-shape parameter for $\Gamma$ distributed rates
                    (if applicable),
                  \item Character substitution rates,
                  \item Base Frequencies.
                \end{itemize}
          \item Find the best root location for the current model
                \begin{enumerate}
                  \item Create a list of high likelihood root locations
                        evaluated at the midpoint of every branch.
                  \item For the top roots (default 5\%), optimize the root
                        location along the branch.
                \end{enumerate}
          \item Repeat from $3(a)$ until a stopping condition is met:
                \begin{itemize}
                  \item The difference between likelihoods between this
                        iteration and the previous iteration is sufficiently
                        small (below user defined parameter \texttt{atol}),
                  \item If early stopping is enabled, the new root location is
                    sufficiently close to the old root location by distance
                    along the branch (below user defined parameter
                    \texttt{brtol}) or,
                  \item More than 500 iterations have passed.
                \end{itemize}
        \end{enumerate}
  \item Report the best found root, along with its log-likelihood
\end{enumerate}

The process for exhaustive mode is analogous, with the core optimization routines
being shared with search mode. The major difference is that all branches are
considered:

\begin{enumerate}
    \item For every branch on the tree:
    \begin{enumerate}
      \item Place root at current branch.
      \item Initialize numerical parameters:
                \begin{itemize}
                  \item $\alpha$-shape parameter for $\Gamma$ distributed rates
                    (if applicable),
                  \item Character substitution rates,
                  \item Base Frequencies.
                \end{itemize}
          \item Optimize model parameters
                \begin{itemize}
                  \item $\alpha$-shape parameter for $\Gamma$ (if applicable),
                  \item Character substitution rates,
                  \item Base Frequencies.
                \end{itemize}
          \item Repeat from $1(c)$ until a stopping condition is met:
                \begin{itemize}
                  \item The difference between likelihoods between this
                        iteration and the previous iteration is sufficiently
                        small (below \texttt{atol}) or,
                  \item If early stopping is enabled, the new root location is
                        sufficiently close to the old root location by distance
                        along the branch (below \texttt{brtol}).
                  \item More than 500 iterations have passed.
                \end{itemize}
    \end{enumerate}
  \item Report the tree with annotations for every branch:
    \begin{itemize}
      \item The root position along the branch,
      \item The log-likelihood,
      \item and the Likelihood Weight Ratio \cite{strimmer_inferring_2002}.
    \end{itemize}
\end{enumerate}

We re-initialize the initial model parameters in every iteration (from (3) in
search mode and (1) in exhaustive mode) to avoid the numerous local minima, as
discussed in \cite{huelsenbeck_inferring_2002}. In both modes, there is an upper
limit to the number of iterations of 500. In empirical and simulated datasets
this limit has never been met, and only exists to ensure that the program will
eventually halt.

\section{The Software}

In order to implement both likelihood computations and non-reversible models,
\RootDiggertt{} has three major dependencies: GNU Scientific Library (GSL)
\cite{gough_gnu_2009}, the Phylogenetic Likelihood Library (LibPLL)
\cite{flouri_phylogenetic_2015}, and L-BFGS-B \cite{zhu_algorithm_1997}.  GSL is
used for the decomposition of a non-symmetric rate matrix, LibPLL is used for
efficient likelihood calculations, and L-BFGS-B is used for multi-parameter
optimizations.

Usage of \RootDiggertt{} is straight-forward. All that is required is a 
tree in newick format, and a MSA in either PHYLIP
or FASTA format. \RootDiggertt{} is open source, released under the MIT license,
and written in C++. The code, documentation, test suite, benchmarks, experiment
scripts, datasets, this paper, as well as any modifications to existing
libraries can be found at \url{github.com/computations/root_digger}.

\section{Experiments and Results}

To validate \RootDiggertt{}, we conducted several experiments on both simulated
and empirical data. Furthermore, we also used Likelihood Weight Ratios (LWR)
\cite{strimmer_inferring_2002} to asses the confidence of root placements on
empirical datasets. Finally, we investigated the effects of the early stop mode on the
final results.

\subsection{Simulations}

Testing with simulated data was done to both validate the software and to
compare against IQ-TREE \cite{minh_iq-tree_2019}.
We created a pipeline to

\begin{enumerate} 
  \item Generate a random tree with ETE3
        \cite{huerta-cepas_ete_2016} and random model parameters.
  \item Simulate an MSA with indelible \cite{fletcher_indelible:_2009}
  \item Execute \RootDiggertt{} and IQ-TREE \cite{minh_iq-tree_2019} with the
    simulated MSA, given the generated random tree.
  \item Repeat from (2) for a total of 100 iterations
  \item Compute comparisons
        \begin{enumerate}
          \item Calculated rooted RF distance with ETE3
                \cite{robinson_comparison_1981}
          \item Mapped root placement onto original tree with the true root.
        \end{enumerate}
\end{enumerate}

Both IQ-TREE and \RootDiggertt{} were given the same model options for all the
runs. \RootDiggertt{} was executed with the arguments

\begin{verbatim}
rd --msa <MSA FILE> --tree <TREE FILE>
\end{verbatim}

By default \RootDiggertt{} uses no rate categories, and currently only supports
the UNREST model \cite{yang_estimating_1994}. IQ-TREE was executed with the
arguments

\begin{verbatim}
iqtree -m 12.12 -s <MSA FILE> -g <TREE FILE>
\end{verbatim}

The \texttt{-m 12.12} argument to IQ-TREE specifies that the UNREST model should
be used \cite{woodhams_new_2015} and the \texttt{-g <TREE FILE>} option
constrains the tree search to the given user tree. When given an fully resolved
unrooted tree, this has the effect of rooting the tree. We used this option to
simulate the operation of \RootDiggertt{}. A notable difference between results
of IQ-TREE and \RootDiggertt{} is that IQ-TREE also will optimize all branch
lengths, and there is no option at the time of writing to turn off this
feature. For that reason, comparison of results between the two tools will be
done using topological distance. Furthermore, the added computational effort
involved in branch length optimization renders execution time comparisons useful
for context only. This is to say, we include execution time comparisons for
reference only.

For all runs, the UNREST model was used. Furthermore, we vary two additional
parameters used for the size of the data in the simulations: number of MSA sites
and taxa.  In total, we ran 9 simulated trials with MSA sizes of 1000, 4000, and
8000 sites as well as tree sizes of 10, 50, and 100 taxa. The results from these
experiment, as well as the execution times, are shown in
figure~\ref{fig:timing-box-plot}. 

%\begin{figure}
%  \begin{center}
%    \includegraphics[width=.9\linewidth]{./figs/sim_results/melted_norm_dist_box.png}
%    \caption{Box plots of normalized topological error.
%    \label{fig:norm_error_boxplot}}
%\end{center}
%\end{figure}

\begin{figure}
  \begin{center}
    \includegraphics[width=\linewidth]{./figs/time_distance_boxplot.png}
    \caption{Comparison of \RootDiggertt{} and IQ-TREE on result error (A) and
      execution time (B). The distance in (A) is the topological distance from
      the inferred root to the true root normalized by the number of taxa on the
      tree.
    \label{fig:timing-box-plot}}
\end{center}
\end{figure}

\begin{figure}
  \begin{center}
    \includegraphics[width=.9\linewidth]{./figs/timing_plots/iq_time_hist.png}
  \caption{Histogram of IQ-TREE times
  \label{fig:iq_time_results}}
\end{center}
\end{figure}

\begin{figure}
  \begin{center}
    \includegraphics[width=.9\linewidth]{./figs/timing_plots/rd_time_hist.png}
    \caption{Histogram of \RootDiggertt{} times
    \label{fig:rd_time_results}}
\end{center}
\end{figure}

\begin{figure}
  \includegraphics{figs/sim_results/100_1000.png}
  \caption{Simulation results for 100 taxa and 1000
  sites. \label{fig:sim-results-100t-1000s}}
\end{figure}

\begin{figure}
  \includegraphics{figs/sim_results/100_8000.png}
  \caption{Simulation results for 100 taxa and 8000
  sites. \label{fig:sim-results-100t-8000s}}
\end{figure}

\begin{figure}
  \includegraphics{figs/sim_results/50_1000.png}
  \caption{Simulation results for 50 taxa and 1000
  sites. \label{fig:sim-results-50t-1000s}}
\end{figure}

\begin{figure}
  \includegraphics{figs/sim_results/50_8000.png}
  \caption{Simulation results for 100 taxa and 8000
  sites. \label{fig:sim-results-50t-8000s}}
\end{figure}

\subsection{Empirical Data}

In addition to simulated data, we conducted tests with empirical data. The
datasets used are described in Table~\ref{table:datasets}.  The empirical
datasets were chosen from TreeBASE \cite{piel2009treebase, vos_nexml_2012} to
include an existing, strongly supported outgroup.  For each of the empirical
datasets, we ran \RootDiggertt{} in exhaustive mode to obtain likelihood weight
ratios (LWR) for each branch.  Annotations are suppressed for branches with a
small LWR (less than $0.0001$).  The trees with annotated LWR are shown in
Figures \ref{fig:spiders-missing-species-no-outgroup} -
\ref{fig:angio-cds12-outgroup} \BenComment{Figures suppressed to save paper}.
We ran the experiments on the datasets with the outgroup included, as well as
with the outgroup removed.

There was some preprocessing performed. In order ensure that all sure branch
lengths in all datasets were specified in substitutions per site, the branch
lengths were re-optimized using RAxML-NG \cite{kozlov_raxml-ng:_2019} version
\texttt{0.9.0git}. The original model was used when it was
known, otherwise the branch lengths were optimized
using GTR+G.

\ctable[
 cap = Empirical Datasets,
 caption = Table of emperical datasets used for
 validation.\label{table:datasets},
 label = table:datasets,
 pos=h,
 ]
{ l c c c c c}{\tnote{The paper states that PartitionFinder was used, but the
  results were not provided.}}{\FL
Dataset & \#Taxa & \#Sites & Model Used in Study & Model Used Here& Source \ML
AngiospermsCDS12 & 35 & 864029 & GTR+$\Gamma$ & GTR + $\Gamma$4 & \cite{ran_phylogenomics_2018}\NN
AngiospermsCDS & 35 & 1296043  & GTR+$\Gamma$ & GTR + $\Gamma$4 &
\cite{ran_phylogenomics_2018}\NN
Grasses & 245 & 4973 &  GTR+ G + I&  GTR + $\Gamma$4 &\cite{christin_molecular_2014}\NN
Ficus & 200 & 5552 & GTRCAT &  GTR + $\Gamma$4 &\cite{cruaud_extreme_2012}\NN
SpidersMissingSpecies & 33 & 1097842 & Not stated in paper\tmark  & GTR +
$\Gamma$4
                      &\cite{leduc-robert_phylogeny_2018}\NN
SpidersMitocondrial & 34 & 12479 & Not stated in paper\tmark &GTR + $\Gamma$4 &\cite{leduc-robert_phylogeny_2018} \LL
}


%\begin{table}
%  \begin{center}
%    \begin{tabular} { l || c c  || c c  || c c || }
   Dataset & \multicolumn{2}{c ||}{Distance} & \multicolumn{2}{c ||}{Topological Distance} &
   \multicolumn{2}{c ||}{Time} \\
           & RD & IQ & RD & IQ & RD & IQ \\
   \hline
   AngiospermsCDS12       & & & & & &\\
   AngiospermsCDS         & & & & & &\\
   Grasses                & & & & & &\\
   Ficus                  & & & & & &\\
   SpidersMissingSpecies  & & & & & &\\
   SpidersMitocondrial    & & & & & &\\
\end{tabular}
\caption{Results for empirical datasets. Distance is the distance from the
estimated root placement to the true root placement, taking into account branch
lengths. Topological Distance is the topological distance from the estimated
root placement to the true root placement. RD Time and IQ time are the run
times}

%    \label{table:emperical_results}
%  \end{center}
%\end{table}

%\begin{figure}
%  \begin{center}
%    \includegraphics[width=.75\linewidth]{
%    ./figs/spiders/missing_species_no_outgroup.png}
%    \caption{SpidersMissingSpecies dataset analyzed without an outgroup.}
%    \label{fig:spiders-missing-species-no-outgroup}
%  \end{center}
%\end{figure}

%\begin{figure}
%  \begin{center}
%    \includegraphics[width=.75\linewidth]{
%    ./figs/spiders/missing_species_with_outgroup.png}
%    \caption{SpidersMissingSpecies dataset analyzed with an outgroup.}
%    \label{fig:spiders-missing-species-outgroup}
%  \end{center}
%\end{figure}

%\begin{figure}
%  \begin{center}
%    \includegraphics[width=.75\linewidth]{./figs/spiders/mito_no_outgroup.png}
%    \caption{SpidersMitocondrial dataset analyzed without an outgroup.}
%    \label{fig:spiders-mito-no-outgroup}
%  \end{center}
%\end{figure}

%\begin{figure}
%  \begin{center}
%    \includegraphics[width=.75\linewidth]{./figs/spiders/mito_with_outgroup.png}
%    \caption{SpidersMitocondrial dataset analyzed with an outgroup.}
%    \label{fig:spiders-mito-outgroup}
%  \end{center}
%\end{figure}

%\begin{figure}[H]
%  \begin{center}
%    \includegraphics[width=.75\linewidth]{figs/ficus/ficus_no_outgroup.png}
%    \caption{Ficus dataset analyzed without an outgroup.}
%    \label{fig:ficus-no-outgroup}
%  \end{center}
%\end{figure}
%
%\begin{figure}[H]
%  \begin{center}
%    \includegraphics[width=.75\linewidth]{figs/ficus/ficus_with_outgroup.png}
%    \caption{Ficus dataset analyzed with an outgroup.}
%    \label{fig:ficus-outgroup}
%  \end{center}
%\end{figure}

%\begin{figure}
%  \begin{center}
%    \includegraphics[width=.5\linewidth]{figs/angio/cds12_no_outgroup.png}
%    \caption{AngiospermsCDS12 dataset analyzed without an outgroup.}
%    \label{fig:angio-cds12-no-outgroup}
%  \end{center}
%\end{figure}
%
%\begin{figure}
%  \begin{center}
%    \includegraphics[width=.5\linewidth]{./figs/angio/cds12_with_outgroup.png}
%    \caption{AngiospermsCDS12 dataset analyzed with an outgroup.}
%    \label{fig:angio-cds12-outgroup}
%  \end{center}
%\end{figure}
%
%\begin{figure}
%  \begin{center}
%    \includegraphics[width=.5\linewidth]{./figs/grasses/plastid245_no_outgroup.png}
%    \caption{Grasses dataset analyzed without an outgroup.}
%    \label{fig:grasses}
%  \end{center}
%\end{figure}

%\begin{figure}[H]
%  \begin{center}
%    \includegraphics[width=.4\linewidth]{figs/spiders/1rate.png}
%    \includegraphics[width=.4\linewidth]{figs/spiders/2rate.png}
%    \caption{SpidersMitocondrial dataset analyzed with varying rate categories.
%    Left is with 1 rate category, and right is with 2.}
%    \label{fig:spiders-rate-compare}
%  \end{center}
%\end{figure}

\subsection{Effect of early stopping on result}

Finally, we investigated the effect of the early stopping criterion on the final
LWR results. To do this, we ran \RootDiggertt{} in exhaustive mode on all
empirical datasets with early stopping enabled and disabled. For most runs, the
results with and without early stopping showed no meaningful (difference in LWR
less than $0.000001$) difference. The dataset that showed the largest difference
in LWR is shown in Figure~\ref{fig:es_mito}. In exchange, the runtime of this
dataset with early stopping enabled is about 1.7 times faster.


%\begin{figure}[H]
%  \begin{center}
%    \includegraphics[width=.4\linewidth]{figs/early_stop_tests/es_test_cds12_no_outgroup.png}
%    \includegraphics[width=.4\linewidth]{figs/early_stop_tests/es_test_cds12_no_outgroup_noes.png}
%    \caption{Effect of early stopping on results. Left is with early stopping,
%    Left is without. Dataset is AngiospermsCDS12}
%    \label{fig:es_angio}
%  \end{center}
%\end{figure}
\begin{figure}[H]
  \begin{center}
    \includegraphics[width=.4\linewidth]{figs/early_stop_tests/es_test_mito_no_outgroup.png}
    \includegraphics[width=.4\linewidth]{figs/early_stop_tests/es_test_mito_no_outgroup_noes.png}
    \caption{Effect of early stopping on results. Left is with early stopping,
    Right is without. Dataset is SpidersMitocondrial and has the largest
  observed difference.}
    \label{fig:es_mito}
  \end{center}
\end{figure}

\section{Discussion}

Compared to IQ-TREE, \RootDiggertt{} performs competitively, as can be seen in
Figure~\ref{fig:timing-box-plot}. The results on simulations are mixed, with
IQ-TREE performing slightly better in terms of root placement in most scenarios.

\RootDiggertt{} was faster on all but the smallest datasets. On the smallest
datasets, IQ-TREE and \RootDiggertt{} performed about equal in terms of
execution time. Again, it should be noted that IQ-TREE {\em does} optimize
branch lengths, while IQ-TREE does not. \RootDiggertt{} being faster than
IQ-TREE is expected. Nonetheless, IQ-TREE is the only tool that will find the
same results as \RootDiggertt{} with comparable methodology. Therefore, the
execution times are only presented here for context.

For empirical datasets \RootDiggertt{} performed well, with it correctly
identifying the root for 3 of the 6 datasets, which is consistent with the
results from simulations. The time required to analyze these datasets ranged
from \~23 minutes (SpidersMitocondrial) to 15 hours (AngiospermsCDS) with early
stopping off.

Comparison with IQ-TREE for empirical results is difficult, because IQ-TREE does
not produce a distribution of likelihood weight ratios. In order to simulate
this to a reasonable degree of confidence, we would need to run IQ-TREE many
times in order to produce a comparable distribution of root positions. Given the
computational cost of analyzing the largest dataset with IQ-TREE, we chose not
to compare with \RootDiggertt{} on empirical datasets.

Qualitatively, \RootDiggertt{} offers other advantages: multiple modes of
inference (search mode and exhaustive mode), as well as the ability to compute
LWR for root positions on a given tree. Again, this can be simulated with many
runs of a more standard phylogenetic analysis tool, but the fidelity and speed
will be significantly impaired. 

\RootDiggertt{} is substantially faster than IQ-TREE, as is shown when comparing
figure~\ref{fig:timing-box-plot}. Furthermore, the accuracy of root placement of
\RootDiggertt{} is nearly the same as IQ-TREE, as also seen in
figure~\ref{fig:timing-box-plot}.  Early stopping proved to be an effective
technique to reduce execution times at an extremely marginal cost in accuracy.
The largest observed difference on empirical datasets between using early stop
mode and is negligent, as it can be seen in figure~\ref{fig:es_mito}. 

In Huelsenbeck \cite{huelsenbeck_inferring_2002}, it was shown that the prior
probability of a root placement on a sample tree did not have a strong signal
when using a non-reversible model of character substitution.  While performing
our verification of \RootDiggertt{} using empirical data, we found that this was
often not the case. For example on the AngiospermsCDS12 dataset (see figure
\ref{fig:angio-cds12-no-outgroup}), we found a clear signal for the root
placement, both with and without the outgroup. Even in cases when the signal was
not as strong, for example SpidersMitocondrial (see figure
\ref{fig:spiders-mito-no-outgroup}), there is a much stronger signal for root
placement than Huelsenbeck would suggest we would get with this kind of analysis
(which is to say, analysis using a non-reversible model). The one exception to
this is the ficus dataset, which showed at least marginal support for the root
on nearly all branches of the tree. We suspect that this is due to Huelsenbeck
performing the analysis on a 4 taxon tree with the distantly related taxa frog,
bird, mouse, and human. By only using 4 distantly related taxa, the rate matrix
is less constrained by the data present, which may lead to over fitting the rate
matrix. We believe that the methods presented here will typically produce
a clear signal for the rooting of a tree.

\section{Conclusion and Future Work}


Going forward with \RootDiggertt{}, there are several developments that would be
useful. One of these is support for additional models. Currently we support only
the most complex model UNREST, but in the future it might be useful to support
less complex models, such as the Lie group models described in
Woodhams~\cite{woodhams_new_2015}.  In particular, models with fewer parameters
are generally regarded as being less prone to over fitting, which might lead to
a better assessment of the true root location.

In addition to more models, other data types could be supported, in particular
amino acid (AA) data. In this work, we decided not to use AA data as it would
increase the number of free parameters from 12 for DNA data to $380$ for AA. We
suspect that this number of parameters is far too prone to over fitting to be
useful, but this has never been investigated.

Finally, there are a few parameters that are not part of the model that could be
heuristically set. These parameters include the number of initial candidate
roots in the search mode and the number of roots to fully optimize during each
step of the search mode. In this work these parameters were performing on well
simulations, but better results could possibly be obtained via an adaptive
strategy.

\bibliographystyle{acm}
\bibliography{main}

\end{document}
