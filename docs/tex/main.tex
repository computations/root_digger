\documentclass{article}

\title{Root Digger: a root placement program for phylogenetic trees}
\author{Ben Bettisworth, Alexandros Stamatakis}
\date{\today}

\begin{document}
\begin{abstract}
In phylogenetic analysis, it is common to infer trees which are unrooted. This
is to say, it is unknown which node is the most recent common ancestor of all
the taxa present in the phylogenetic tree. This information is often desirable,
as it provides a direction to the edges in the tree.  There exist several
methods to recover a root, including midpoint rooting and using a special taxon
as a so called outgroup. Additionally, a non-reversable Markov model can be used
to compute the likelihood of a root. In this paper, we present software which
uses a provided non-reversable model to compute the most likely root location.
\end{abstract}

\maketitle

% Main Points:
% - Rerunning analysis with outgroups is expensive, and can introduce errors
% - Midpoint rooting is a joke
% - This is fast and accurate.


\section{Introduction}

% Outline:
% - What is rooting
% - Why do we need it
% - Existing methods
% - What this method is
% - Why should we use this method

In phylogenetic analysis, it is common to infer unrooted trees. This is due in
part to the simplicity of the models, and the computational savings that come
from working with unrooted trees. But a having a root on a phylogenetic tree is
a desirable thing to have. Many phylogenetic studies root the tree after
inferring an unrooted tree.

There are several methods to root an existing phylogenetic tree, which fall into
one of three categories: methods which use "side-band" information; methods
which utilize some variation of the molecular clock hypothesis; and methods
which use a non-reversable model.

Methods that use "side-band" information take advantage of prior knowledge about
the world, which isn't present in the data that is used to infer a tree. For
example, knowledge about which species are distantly related can be used to add
an outgroup to phylogenetic analysis. This outgroup can then be used to place
the root on the tree.

There are challenges to adding  an  outgroup  to  an  analysis.   Gatsey  et.al.
\cite{gatesy_how_2007} showed that adding a  single  taxa  to  an  analysis  can
significantly impact the stability of the tree topology, even for the taxa which
were already present  in  the  analysis.   Holland  \cite{holland_outgroup_2003}
investigates this phenomenon in simulations, and that  outgroups  effecting  the
topology of ingroups can be uncomfortably common.

Alternativly, molecular clock analysis can be used to place a root without prior
knowledge. The molecular clock hypothesis says that base substitution "ticks" at
stochastically constant rate. Using this assumption, or some variation of it, a
likely root can be placed on the root.

Molecular clock analysis can also have difficulties. 

To begin with, some definitions: A graph is a collection of objects (vertices)
and their relationships (edges). A tree is a specific type of graph with exactly
one path between any two nodes. A phylogenetic tree is a tree with verticies
representing species, and edges representing an evolutionary process. Thus, a
phylogenetic tree is a representation of evolutionary history and relationships. 

Generally, phylogenetic trees have an additional constraint: every node is
either trivalent, or univalent. This is to say, every vertex is a leaf,
sometimes called a tip, or an interior node with exactly 3 adjacent vertices.
The exception to this is a specially designated node called the root. The root
is a bivalent vertex, i.e. it has two adjacent vertices. These vertices are the
children of the root.

In addition to a phylogenetic tree, phylogenetic inference requires a model of
evolution. Commonly, a Markov substution process is used
\cite{yang_computational_2006}, along with molecular data such as DNA sequences.
Using this model of evolution, a probability of one species evolving into
another in a specified amount of time can be calcluated. By representing each
edge of the tree as one of these Markov processes, we can calculate the
liklihood of both the subsutution processes, and the choosen topology of the
tree.

Phylogenetic inference often does not seek to find a root. The reasons for this
are many, but a large reason is the pully principle
\cite{felsenstein_evolutionary_1981}. To summarize, the pully priniciple shows
that on a phylogenetic model with a reversable markov process, the root
placement doesn't affect the final likelihood of the tree.

\section{The Software}

% Outline:
% - Options
% - How the software works (maybe)

\section{Experiment and Results}

% Experiments:
% - Run tests on 2 dimensions: Taxa and Sites
% - Taxa: 10, 100, 1,000, 10,000
% - Sites: 100, 1,000, 10,000, 100,000
% - Ultrametric versions?

% Outline:
% - indel data generation
% - run setup (cpu, memory etc)
% - show results (time, accuracy)
%   - Accuracy should be measured by getting the root on the right edge, as well
%   as how far along that edge the root is.

\section{Conclusion}

% Outline:
% - Tell 'em what you told 'em

\bibliographystyle{acm}
\bibliography{main}

\end{document}
